\documentclass[11pt]{article}
\usepackage{amsmath,amssymb,amsthm}
\usepackage{geometry}
\geometry{margin=1in}

\newtheorem{theorem}{Theorem}[section]
\theoremstyle{definition}
\newtheorem{definition}[theorem]{Definition}

\begin{document}

\title{Existence of a Riemannian Metric (atlas + partition of unity)}
\author{}
\date{}
\maketitle

\begin{definition}[Partition of unity subordinate to a cover]
Let \(M\) be a smooth manifold and let \(\{U_i\}_{i\in\mathcal I}\) be an open cover of \(M\).
A \emph{smooth partition of unity on \(M\) subordinate to the cover \(\{U_i\}_{i\in\mathcal I}\)} is a locally finite family
\(\{\rho_i\}_{i\in\mathcal I}\) of smooth functions \(\rho_i:M\to[0,1]\) such that
\begin{enumerate}
  \item \(\operatorname{supp}(\rho_i)\subset U_i\) for every \(i\in\mathcal I\),
  \item for every \(p\in M\) the sum \(\sum_{i\in\mathcal I}\rho_i(p)=1\) (the sum is finite at each point by local finiteness).
\end{enumerate}
\end{definition}

\begin{theorem}
Let \(M\) be a smooth \(n\)-dimensional manifold. Assume there exists a smooth partition of unity subordinate to any given open cover of \(M\).
Let \(\{(U_i,\varphi_i)\}_{i\in\mathcal I}\) be an atlas of coordinate charts covering \(M\), and let \(\{\rho_i\}_{i\in\mathcal I}\) be a smooth partition of unity subordinate to the cover \(\{U_i\}_{i\in\mathcal I}\).
Then \(M\) admits a smooth Riemannian metric \(g\); i.e. a smooth section
\[
g\in\Gamma(M,\,T^*M\otimes T^*M)
\]
whose fibrewise bilinear forms \(g_p\) are symmetric and positive-definite.
\end{theorem}

\begin{proof}
For each index \(i\) define a local \((0,2)\)-tensor \(g^{(i)}\) on \(U_i\) by pulling back the Euclidean inner product on \(\mathbb{R}^n\) along the chart map \(\varphi_i\):
for \(p\in U_i\) and \(v,w\in T_pM\),
\[
g^{(i)}_p(v,w)=\big\langle d(\varphi_i)_p(v),\,d(\varphi_i)_p(w)\big\rangle_{\mathrm{Euc}}.
\]
Each \(g^{(i)}\) is smooth on \(U_i\), symmetric, and positive-definite on each fibre \(T_pM\).

Now use the given partition of unity \(\{\rho_i\}_{i\in\mathcal I}\) subordinate to the cover \(\{U_i\}\).
Define a global \((0,2)\)-tensor field \(g\) on \(M\) by the locally finite sum
\[
g \;:=\; \sum_{i\in\mathcal I} \rho_i\, g^{(i)}.
\]
Concretely, for \(p\in M\) and \(v,w\in T_pM\),
\[
g_p(v,w)=\sum_{i\in\mathcal I} \rho_i(p)\, g^{(i)}_p(v,w).
\]
Local finiteness of the partition ensures the sum is finite at each point, so \(g_p\) is well-defined. Each summand \(\rho_i g^{(i)}\) is a smooth section with support in \(U_i\), and near any point only finitely many summands are nonzero; hence the coordinate components of \(g\) are finite sums of smooth functions and therefore smooth. Thus \(g\in\Gamma(M,T^*M\otimes T^*M)\).

Symmetry is immediate from symmetry of each \(g^{(i)}\) and the scalar factors \(\rho_i\).

To check positive-definiteness: fix \(p\in M\) and a nonzero \(v\in T_pM\). Since \(\sum_i\rho_i(p)=1\) and each \(\rho_i(p)\ge0\), there is at least one index \(i\) with \(\rho_i(p)>0\). For such an \(i\), because \(g^{(i)}_p\) is positive-definite, \(g^{(i)}_p(v,v)>0\). Every term \(\rho_j(p)g^{(j)}_p(v,v)\) is \(\ge0\), and at least one is \(>0\); therefore
\[
g_p(v,v)=\sum_j \rho_j(p)g^{(j)}_p(v,v)>0.
\]
Thus \(g_p\) is positive-definite for every \(p\in M\), hence non-degenerate. Therefore \(g\) is a Riemannian metric on \(M\).
\end{proof}

\end{document}
